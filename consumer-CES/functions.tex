


%%%%% CES 

% indifference curve 
\tikzset{declare function = {indiffCES(\U,\alpha,\rho,\x) = (  ( (\U^\rho) - (\alpha*(\x^\rho)) ) / (1-\alpha)  )^(1/\rho);}}

% max x-domain 
\tikzset{declare function = {xMaxCES(\U,\alpha,\s) = ((\U^\s)/\alpha)^(1/\s);}}

\tikzset{declare function = {denom(\px,\py,\alpha,\s) =  ((\alpha^\s)*(\px^(1-\s))) + (((1-\alpha)^\s)*(\py^(1-\s))) ;}}

% u-maxing x
\tikzset{declare function = {xStar(\px,\py,\alpha,\s,\M) = \M * ((\alpha/\px)^\s) / denom(\px,\py,\alpha,\s);}}

% u-maxing y
\tikzset{declare function = {yStar(\px,\py,\alpha,\s,\M) = \M * (((1-\alpha)/\py)^\s) / denom(\px,\py,\alpha,\s);}}

% price index (useful shortcut for value function and expenditure function)
\tikzset{declare function = {priceIndex(\px,\py,\alpha,\s) = denom(\px,\py,\alpha,\s)^(1/(1-\s));}}

% value function 
\tikzset{declare function = {value(\px,\py,\alpha,\s,\M) = \M/priceIndex(\px,\py,\alpha,\s);}}

% expenditure function 
\tikzset{declare function = {expenditure(\px,\py,\alpha,\s,\U) = \U*priceIndex(\px,\py,\alpha,\s);}}

% find new price px to achieve given utility
\tikzset{declare function = {newpx(\py,\alpha,\s,\U,\M) = (((\U/\M)^(\s-1) - ((1-\alpha)^\s)*(\py^(1-\s)))/(\alpha^\s))^(1/(1-\s));}}

%

